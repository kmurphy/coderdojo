% !TEX TS-program = LuaLaTeX

\documentclass{coderdojo}

\usepackage[lf,sfdefault]{gandhi}

\def\SRC{../resources/docs/Coding_Games_in_Python__DK__2018.pdf}
\def\MUcode{/Users/kmurphy/mu_code/coderdojo_tramore/fruit_ninja_games/}

\newfontfamily{\pygameZeroFont}{Marker Felt}
\def\pygameZero{{\pygameZeroFont Pygame Zero}}

\usepackage{pdflscape}

\usetikzlibrary{decorations,decorations.pathreplacing,decorations.pathmorphing}
\tikzstyle{postit}=[fill=yellow!50,draw,thick,
decorate, drop shadow,
decoration={random steps,segment length=3pt,amplitude=1pt},
text width=4cm, font=\scriptsize]

\worksheet{22}{Collector Games}

\newcommand\contentsitem[2]{
	\item \hyperref[#1]{\color{section}\bfseries #2}
}

\usepackage{wrapfig}
\usepackage{float}

\newcommand\TODO[1]{
\begin{itemize}
\item[\todoSymbol] \color{todo} #1
\end{itemize}}


\newcommand\TEST[1][\bf Test your code!]{
	\centerline{\tikz\node[starburst, fill=yellow, draw=red, line width=2pt,align=center] {#1};}
}

\newcommand\TESTSMALL[2][\bf Test your code!]{
{\tikz[scale=#2]\node[starburst, fill=yellow, draw=red, line width=2pt,align=center] {#1};}
}

\usetikzlibrary{decorations.pathreplacing}

%: DOCUMENT
\begin{document}
\maketitle

\section*{Introduction}

Collector games are based around the idea of the player moving around the screen trying to collect coins as quickly as possible.

As in our fruit ninja game we will develop multiple versions as we add more and more features. The features for this game are similar to what you did for the Fruit Ninja games to test how well you understood the code you implemented there.

\begin{dingautolist}{192} 
\contentsitem{ball}{Basic coin collector game}%
\quad\dotfill\quad\code{collector_basic.py}

Our starting version of the game consists of coins appearing randomly on the screen and the player has to move the character to collect the coins before the limited time runs out.

The steps given here come from the excellent \href{https://www.dk.com/uk/book/9780241317792-computer-coding-python-games-for-kids/}{\em Coding Games with Python} book.

\contentsitem{rectangle}{Adding sounds}%
\quad\dotfill\quad\code{collector_sound.py}

Go to \href{https://www.zapsplat.com}{www.zapsplat.com}, and find some sounds that we can use\\ when you collect a coin, perhaps a clot-machine ring would work. 
\hfill\raisebox{-12pt}[0pt][0pt]{\href{https://www.zapsplat.com}{\includegraphics[height=30pt]{zap-splat-logo}}}

\parbox[t]{11cm}{For background music have a look at 
\href{https://www.melodyloops.com/music/}{www.melodyloops.com}\\
This has a good selection of tracks and can cut a track to whatever length you want --- you could set the length to match the time given to complete the level.}
\hfill\raisebox{-18pt}[0pt][0pt]{\href{https://www.melodyloops.com/music/}{{\includegraphics[height=20pt]{melodyloops}}}}


\contentsitem{rectangle}{Making the game more playable}%
\quad\dotfill\quad\code{collector_levels.py}

Currently the game just runs down a clock and your score is how many coins selected. The problem here is that there is no real goal --- other than collecting as many as possible. A nicer version would be to have multiple levels, and in each level the player must pick up, say 5 coins to win, but the time allocated gets shorter and shorter.
  
\contentsitem{rectangle}{Varying targets}%
\quad\dotfill\quad\code{collector_targets.py}

We could have multiple targets --- some of which are worth more than others. 

\contentsitem{rectangle}{Disappearing targets}%
\quad\dotfill\quad\code{collector_vanishing_targets.py}

We could have targets that disappear if they are not picked within a (small) time interval.

\end{dingautolist}

\clearpage

%: PAGES
\renewcommand{\baselinestretch}{0.55} 


\newcommand\PAIR[3]{
\begin{landscape}\thispagestyle{empty}\hspace{-1.25cm}
\begin{tikzpicture}
\useasboundingbox (0,0) rectangle (28.1,16.75);

\ifnum#1=0\else
\node[anchor=south west] at (0,0) (L) {\includegraphics[page=#1, scale=0.707,angle=0]{\SRC}};\fi
\ifnum#2=0\else
	\node[anchor=south west] at (L.south east) (R)
	{\includegraphics[page=#2,scale=0.707,angle=0]{\SRC}};
	\fi

\node at ($(L.north east)+(0,0.5)$) {\ding{192} \color{section}\bfseries collector\_basic};
#3
\end{tikzpicture}
\end{landscape}}

\def\GRID#1{
\draw (#1.south west) grid (#1.north east);
\foreach \x in {0,1,...,13} {
	\node at ($(#1.south west) + (\x,0.5) $) {\scriptsize\x}; 
	\node at ($(#1.south west) + (\x,8.5) $) {\scriptsize\x}; 
	\node at ($(#1.south west) + (\x,16.5) $) {\scriptsize\x}; 
}
\foreach \y in {1,...,16} {
	\node at ($(#1.south west) + (0, \y) $) {\scriptsize\y}; 
	\node at ($(#1.south west) + (13, \y) $) {\scriptsize\y}; 
}}


\tikzstyle{textfix}=[fill=white,anchor=south west,inner sep=0pt,outer sep=0pt,font=\textfixFont]

\def\BUTTON#1{\includegraphics[width=12pt]{#1}}

%: PAIR{62}{63}
\PAIR{62}{63}{
	\node[textfix, anchor=north west,text width=4.5cm,text depth=2cm] at ($(R.south west) + (1.7,4.15)$) {
	};
	
	\node[textfix, anchor=north west,text width=12.5cm] at ($(R.south west) + (1.7,3.6)$) {{\bf Setup Game}\\ In the  mu-editor, 
	\\$\bullet$ Click on the new file button, \BUTTON{new}
	\\$\bullet$ Click on the save file button, \BUTTON{save};	
	\\[6pt]$\bullet$ Change folder to {\smaller\tt coderdojo\_tramore}
	\\[6pt]$\bullet$ Create a new folder called {\smaller\tt collector\_games}
	\\[6pt]$\bullet$ Save game in new folder, using name {\smaller\tt collector\_basic}
	\\[20pt]};
}


%: PAIR{64}{65}
\PAIR{64}{65}{

%\GRID{L}\GRID{R}

% step 2
\node[textfix,text width = 5cm,anchor=north west] at 
	($(L.south west) + (1.7, 15)$) {
	This game uses two images --- a fox and a coin. Within your  
	collector\_games folder, we need to create a new folder called images. 
	The Mu editor will do this for us, when you click on the images button,
	\BUTTON{images}. \\
	
	\rule{0pt}{3cm}};

% step 3
\node[textfix,bottom color=gray!25,top color=gray!10] at 
	($(L.south west) + (9.6, 13.1)$) {\smaller collector\_games};
\node[textfix,bottom color=gray!20,top color=gray!20] at 
	($(L.south west) + (7.8,12.74)$) {\smaller collector\_basic.py};


% step 9
\node[textfix,text width = 6cm,anchor=north west] at 
	($(R.south west) + (6, 5)$) {
	\\[20pt]\mbox{}};
\node[textfix,text width = 5cm,anchor=north west] at 
	($(R.south west) + (1.7, 4.5)$) {
	Now test the code you've written so far.\\[20pt]\mbox{}};
\node[textfix,text width = 3cm,anchor=north west] at 
	($(R.south west) + (8, 4)$) {
	\\[45pt]\mbox{}};
}

%: PAIR{66}{67}
\PAIR{66}{67}{
}

%%: PAIR{68}{69}
%\renewcommand{\baselinestretch}{0.55} 
\PAIR{68}{69}{
}

\end{document}